\documentclass[11pt, oneside]{article}   	% use "amsart" instead of "article" for AMSLaTeX format
\usepackage{geometry}                		% See geometry.pdf to learn the layout options. There are lots.
\geometry{letterpaper}                   		% ... or a4paper or a5paper or ... 
%\geometry{landscape}                		% Activate for rotated page geometry
%\usepackage[parfill]{parskip}    		% Activate to begin paragraphs with an empty line rather than an indent
\usepackage{graphicx}				% Use pdf, png, jpg, or eps§ with pdflatex; use eps in DVI mode
								% TeX will automatically convert eps --> pdf in pdflatex		
\usepackage{amssymb}

%SetFonts

%SetFonts


\title{Visual Communication in Context}
%\author{The Author}
\date{}							%' Activate to display a given date or no date

\begin{document}
\maketitle
\section{Introduction}

Communication is not limited to verbal language; humans can make use of many different tools to convey meaning to a partner. While theories of communication often rely upon the existence of modality-general pragmatic reasoning mechanisms, these predictions have rarely been tested outside the verbal modality. In this experiment, we aim to test how context -- the set of potential referents in common ground -- affects visual communication: using drawings to convey the identity of rich naturalistic images. Past work in the verbal modality has shown a strong influence of speaker informativity: when the intended referent is surrounded by very similar objects, speakers send more detailed and verbose messages to help the listener distinguish it. We expect analogous results to hold in the visual modality. For instance, when referring to a car in a context containing dogs and birds, speakers should use simpler, more abstract drawings than when referring to the same object in the context of other cars.

\section{Methods}

Participants are paired in an online environment to play an interactive reference game. Both players are shown an array of four objects (in randomized order). One player is assigned to the role of sketcher and privately sees one of the four objects highlighted to designate it as the `target.' Their goal is to produce a sketch such that their partner -- the viewer -- can correctly click on the target. Sketchers use a small sketchpad to make drawings, which are rendered upon completion of each stroke on the viewer's screen. When the sketcher is finished drawing, they click a button which notifies the viewer that it is their turn to click. The critical manipulation 

\section{Results}

We will collect stroke-by-stroke svg strings produced by the sketcher on every trial, as well as the viewer's selection among the objects. We operationalize the notion of `simplicity' through a small set of derived stroke-level measures. First, we expect the raw number of strokes per trial to be higher in the close condition than the far condition. Second, we expect the the length of the svg string to be higher in the close condition than the far condition, which should be correlated with raw number of strokes but accounts for the complexity of each stroke. Finally, we will conduct a range of model-based analyses using features extracted from modern convolutional neural networks to rigorously evaluate pragmatic and non-pragmatic accounts of communication under different assumptions about how the visual system computes similarity. 

\end{document}  