\documentclass{evolang12}
\usepackage{graphicx}
\usepackage{url}
\usepackage{authblk}


\begin{document}


\title{Lexicalization of conceptual hierarchies \\ through interaction}
\author[1]{ANONYMOUS AUTHOR 1}
\author[1]{ANONYMOUS AUTHOR 2} 
\author[2]{ANONYMOUS AUTHOR 3}
\author[3]{ANONYMOUS AUTHOR 4}
\affil[*]{Corresponding Author: name@domain.com}
\affil[1]{This Department, University X, City, Country}
\affil[2]{That Department, University Y, City, Country}
\affil[3]{Other Department, University Z, City, Country}

%%%%% INSTRUCTIONS FOR ADDING AUTHORS %%%%%

%Initial submissions should be anonymous - DO NOT INCLUDE ANY IDENTIFYING INFORMAITON IN THE SUBMISSION VERSION -(i.e., no author details should be added above). To avoid problems with page limits, please include the appropriate number of placeholders for authors. If multiple authors share an affiliation, they can be paired with the appropriate affiliation using the numbers in the square brackets next to "author" and "affil" to save space. Designate a single corresponding author using the asterisk.

\maketitle

Natural languages provide speakers with remarkable flexibility in their choice of referring expressions. In addition to the combinatorial explosion of possible modifiers afforded by compositional semantics \cite{Partee95_LexicalSemanticsCompositionality,VanDeemterEtAl12_ReferenceProduction,WintersKirbySmith14_LanguagesAdapt}, we have a number of lexicalized nominal terms at our disposal. \emph{Dalmatian}, \emph{dog}, and \emph{animal} can all truthfully be used to talk about the same Dalmatian at different levels of specificity, with one level privileged over the others --- the \emph{basic-level} \cite{RoschMervisGray___BoyesBraem76_BasicObjects}. While recent work has investigated the pragmatics of which terms speakers prefer in different contexts \cite{GrafEtAl16_BasicLevel}, there remains a more fundamental evolutionary question about why so many levels of reference became lexicalized and why a basic-level exists in the first place.

Our hypothesis, motivated both by classic work on concept representations and contemporary work on the selective pressures induced by communication, is that lexicalization of conceptual hierarchies is a function of (1) the overall statistics of the environment (e.g. which features tend to cluster together) and (2) the particular contexts in which communication occurs (e.g. which objects are relevant to distinguish). To test this hypothesis, we designed an repeated reference game in which a pair of participants must collaboratively create a language through interaction. 

\bibliographystyle{apacite}
\bibliography{evolang} 

\end{document}
